\documentclass[pre,twocolumn,twoside,superscriptaddress,floatfix, aps, 10pt]{revtex4-1}

\usepackage{currfile}
\lefthyphenmin=3
\righthyphenmin=2

\usepackage{graphicx,epsfig,verbatim,enumerate}
\usepackage{amssymb,amsmath}
\usepackage{ifthen}

% Fonts and typesetting settings
\usepackage[sc]{mathpazo}
\usepackage[T1]{fontenc}
\linespread{1.05} % Palatino needs more space between lines
\usepackage{microtype}
\usepackage{textcomp}

\usepackage{longtable}


\usepackage{mathtools}

\newboolean{twocolswitch}

\newcommand{\www}[1]{\url{#1}}
\newcommand{\req}[1]{(\ref{#1})}
\newcommand{\Req}[1]{Eq.~(\ref{#1})}

% Lettrines
\usepackage{lettrine}

%useful shortcuts
\def\R{\ensuremath{\mathbb{R}}} %\ensuremath adds math mode, if forgotten
\def\Q{\ensuremath{\mathbb{Q}}}
\def\N{\ensuremath{\mathbb{N}}}
\def\Z{\ensuremath{\mathbb{Z}}}
\def\C{\ensuremath{\mathbb{C}}}

%shorcuts with arguments
\newcommand{\abs}[1]{\left\vert#1\right\vert} %nice absolute values
\newcommand{\bt}[1]{\textbf{#1}} %bold
\newcommand{\eq}[1]{\begin{align*}#1\end{align*}} %aligned equations
\newcommand{\norm}[1]{\left\lVert#1\right\rVert} %vector norm
\renewcommand{\eq}[1]{\begin{align*}#1\end{align*}} %aligned equations

%piecewise function

%\begin{displaymath}
%   f(x) = \left\{
%     \begin{array}{lr}
%       1 & : x \in \mathbb{Q}\\
%       0 & : x \notin \mathbb{Q}
%     \end{array}
%   \right.
%\end{displaymath} 


%environment
\newcommand{\tab}{\phantom{ssss}}

\usepackage{hyperref}
\usepackage{color}
\newcommand{\todo}[1]{\noindent\textcolor{blue}{{$\Box$ #1}}}

%colors
\definecolor{javagreen}{rgb}{0.25,0.5,0.35} %dark green color
\definecolor{lightblue}{rgb}{0.149,0.545,0.824} %solarized blue
\definecolor{sred}{rgb}{0.863, 0.196, 0.184} %solarized red

\newcommand{\blue}[1]{{\leavevmode\color{lightblue}{#1}}} %solarized blue 
\newcommand{\green}[1]{{\leavevmode\color{javagreen}{#1}}} %command for green
\newcommand{\red}[1]{{\leavevmode\color{sred}{#1}}} %solarized red
\newcommand{\gray}[1]{{\leavevmode\color[gray]{0.5}{#1}}} %gray text
\newcommand{\Prob}[1]{{\rm Pr}\left(#1\right)}


\setboolean{twocolswitch}{true}


\begin{document}

\title{\protect
Connecting every bit of knowledge: \\
The Structure of Wikipedia's first link network
}

\date{\today}

\author{
\firstname{Mark}
\surname{Ibrahim}
}
\email{mark.s.ibrahim@uvm.edu}

\affiliation{Department of Mathematics \& Statistics, 
    Computational Story Lab,  \\
    Vermont Complex Systems Center,
    Vermont Advanced Computing Core,
    The University of Vermont, Burlington, VT 05401.}

\author{
\firstname{Christopher}
\surname{M. Danforth}
}
\email{chris.danforth@uvm.edu}

\affiliation{Department of Mathematics \& Statistics, 
    Computational Story Lab,  \\
    Vermont Complex Systems Center,
    Vermont Advanced Computing Core,
    The University of Vermont, Burlington, VT 05401.}

\author{
\firstname{Peter}
\surname{Sheridan Dodds}
}

\email{peter.dodds@uvm.edu}

\affiliation{Department of Mathematics \& Statistics, 
    Computational Story Lab,  \\
    Vermont Complex Systems Center,
    Vermont Advanced Computing Core,
    The University of Vermont, Burlington, VT 05401.}


\begin{abstract}
  \protect
Apples, porcupines, and the most obscure Dylan song---is every topic a few clicks from Philosophy? 
Within Wikipedia, the surprising answer is yes: nearly all 
paths lead to Philosophy.
Wikipedia is the largest, most meticulously indexed collection of human knowledge ever amassed. 
More than information about a topic, Wikipedia is a web of naturally emerging relationships.  
By following the first link in an article, we connect entries to form a directed network: Wikipedia's First Link Network. 
Here, we study the English edition of Wikipedia's First Link Network for insight into how the many topics, ideas, people, objects, and events
are related and organized.  
We algorithmically parse all 4.7 million articles to construct a map of Wikipedia's First Link Network. 
We traverse every possible path through the network, 
measuring the accumulation of first links, path lengths, basins, cycles, and the influence a particular article exerts in shaping the 
network.
We discover many scale-free distributions describing path length, 
accumulation, and influence; find Philosophy at a salient center; and uncover a flow from specific to general 
culminating around fundamental notions such as Community, State, and Science. 
Curiously, we also observe a gravitation towards topical articles including Health Care and Fossil Fuel.
These findings enrich our view of the connections and structure of
Wikipedia's ever growing store of knowledge.
 
\end{abstract}

\maketitle

\section{Introduction}

Wikipedia is a towering achievement of the modern era. 
At no point in history has a larger or more meticulously indexed collection of human knowledge 
existed.
Wikipedia contains 37 million articles in 283 languages, 
with coverage spanning everything from little known ancient battles to the latest pharmaceutical drugs 
\cite{drugs, stats}.
All this information does not lie dormant. 
Wikipedia is the sixth most visited
site in the world, surpassing 18 billion page views and 26 million edits in a single month
\cite{wiki_edits, wiki_views}.

Wikipedia has naturally become the object of many studies. 
Researchers have examined the cultural dynamics among editors
\cite{editors},
the accuracy of the content relative to traditional encyclopedias
\cite{accuracy1, accuracy2},
the topics covered 
\cite{coverage},
and bias against portions of the population
\cite{bias_women}.
Wikipedia's content has also proven a powerful tool. 
Researchers have used Wikipedia to identify missing dictionary entries
\cite{missing_entries},
cluster short text
\cite{clustering},
compute semantic relatedness
\cite{semantic_relatedness},
and disambiguate meaning \cite{disambiguating}.

While these many studies have dissected and fruitfully applied Wikipedia's content,
few have examined the connections among the many articles
\cite{links_that_speak}.
A hyperlink from one Wikipedia article to another
naturally indicates a relationship between the two articles
\cite{relevance}.
The notion that hyperlinks convey information about the content of a
page has proved enormously successful in multiple domains from search engine algorithms 
such as PageRank 
\cite{pagerank} 
to topic classification
\cite{classifier}.
We treat a hyperlink as a mechanism connecting two topics.

The authors of a Wikipedia article choose where and whether to include a 
reference to another Wikipedia article
in the HTML markup.
For example, the authors of the 
\href{https://en.wikipedia.org/wiki/Train}{``Train''}
article chose
``Rail Transport,'' ``Cargo,'' and ``Passengers'' 
as pertinent articles to reference in describing trains
\cite{wiki_train}
.
Here we focus on the first hyperlink to another Wikipedia article, which 
marks the earliest moment in the body text where the authors
chose to connect the information discussed to that of another article.
By following the first hyperlink to another article in the English edition of
Wikipedia as of November, 2014, we connect the ideas in one article to those in another, ultimately forming a directed network: 
{\it Wikipedia's First Link Network} (FLN).

\begin{figure*}[tp!]
  \includegraphics[width=\textwidth]{graphics/Train.pdf}  
  \caption{
    \textbf{First Link Path For ``Train.''}
    We follow the First Link to another Wikipedia article
    in the main body of the article---the area
    inside the green rectangle, which excludes 
    side bar elements, the navigation bar and title. 
    In this example, the First Link to another Wikipedia article is ``Rail Transport.'' 
    We can again select the First Link on the ``Rail Transport''
    article, repeating the process to 
    form a path of First Links.
  }
  \label{fig:Train First Links}
\end{figure*}

What information could the First Links possibly reveal?
Inspired by the claim that the majority of First Links lead to the 
``Philosophy'' article---popularized by an 
\href{https://xkcd.com/903/}{xkcd}
comic and subsequently
discussed in blog posts 
\cite{mat_blog, Ilmari_first_links, xkcd}
---we holistically study 
Wikipedia's FLN as a map relating areas of human knowledge. 
The map is a hierarchical taxonomy where, for example, ``Train'' links to a parent node, 
``Rail Transport,'' while many child nodes, ``Steel'' and ``Horsepower,'' link to 
``Train.'' Unlike previous taxonomies
created by individuals
\cite{locke, descartes, aristotle}
or a select group of individuals 
\cite{hist_thesaurus}, 
the organization of ideas in the FLN 
does not emerge from a centralized effort. 
Rather each link in the FLN emerges from the choice of each author and ensuing edits.
Collectively, the FLN is a wealth of relations among inventions, places,
figures, objects, and events across space and time.


Our goal is to study the structure of the FLN for insight into how the information on Wikipedia is organized and related.
We develop metrics to capture 
the dominant features of the FLN's structure.
We measure dynamics of the FLN as a flow, quantifying 
aspects of the FLN from the accumulation of First Links around articles 
to the influence an article exerts in shaping the FLN.
Together with cycles, in-degree, depth, and the content of the articles, 
we build our analysis of the relations among the ideas in Wikipedia.



\section{Traversing the First Link Network}

An essential feature of a directed network's structure is the degree distribution 
\cite{newman}. 
The degree distribution has been used to study many phenomena from disease outbreak 
\cite{disease} 
to the dynamics of social networks 
\cite{social_nets}.
The in-degree distribution in the FLN describes how many first links point to a 
particular article. 
We measure the in-degree in the FLN to gauge the number of articles referencing a particular article.
Articles with zero in-degree have no references---they are outer leaves in the FLN. 
Similar to PageRank, the in-degree 
provides a way to rank the articles in Wikipedia, allowing us to see 
whether authors tend to reference all articles equally or only a few.


Our greater interest is lies in the richer dynamics of how many articles fit together---how links create a flow through the FLN.
We construct a path through the FLN with each article 
leading to the next through the first link. 
For example ``Train'' has a first link to ``Conveyance of Passengers and Goods,'' which is itself
an article with a first link to ``Goods'' and so on. 
As shown in Fig.~\ref{fig:Train First Links}, the path starting at ``Train'' 
contains ``Goods,'' ``Economics,'' ``Social Science,''
leading ultimately to ``Philosophy''.

\begin{figure*}[tp!]
  \includegraphics[width=\textwidth]{graphics/traversal_visit_algo_figure.pdf}  
  \caption{
    \textbf{Traversal Visit Algorithm on a sample network.}
     The traversal visit vectors are an adjacency matrix for the paths through the network: the first column is indicates the path formed starting with article A. The number of traversal visits for article A is then the number of paths containing A or the sum of the first row in our matrix:
     $\sum_{i=1}^7 A_{\text{visit, i}} = 7$.
  }
  \label{fig:Traversal Visits}
\end{figure*}

Starting at each article, we construct a path through the FLN, 
to map the flow of connections among articles.
The method is order agnostic with respect to which articles are selected first. As long as each article is selected eventually, the resulting metrics are equivalent.

We use the notion of a path through the FLN to describe 
the structure of the FLN as a flow relating topics. 
Grounded in the dynamics of river networks
\cite{geo_basins},
we characterize features of the FLN as a flow through first links.
Previous studies have used flow to characterize the structure of river networks
\cite{dodds} and describe the organization of food systems through transportation networks
\cite{food_webs}.
With paths to mark flow through the FLN, we develop metrics to 
gauge the accumulation of references, 
the length of the path relating articles, and the influence a particular
article exerts in shaping the flow of links through the FLN.


The first metric we develop quantifies the accumulation of first links.
The algorithm begins by selecting an article, then traversing the path formed
by following the first links. Each time a first link references an article, we increment a count
associated with the article. 
We continue until the first link is retraced or is invalid---defined here to mean outside of Wikipedia.
We select a second article and repeat the process until we have 
constructed a path for each article in the network. We call the resulting count for each article the number of {\it traversal visits}. The number of traversal visits of an article 
measures the number of references flowing to the ideas in the article---equivalent
to basin area in geomorphology. 

\begin{figure}[tp!]
  \includegraphics[width=\columnwidth]{graphics/traversal_funnel_algo_figure.pdf}
  \caption{
    \textbf{Traversal Funnel Algorithm on a sample network.}
  The algorithm for traversal funnels is identical to the previous algorithm for traversal visits with one alteration: the path ends at the start of a cycle to distinguish articles directing a path into a cycle from articles that simply happen to be in a highly traversed path. We can construct similar vectors by considering each path through the network, measuring traversal funnels for a particular article as the sum of the entries in its corresponding row. For example
  the number of traversal funnels for article $E$ is 
  $\sum_{i=1}^7 E_{\text{funnel, i}} = 2$.}
  \label{fig:Traversal Funnels}

\end{figure}

We can characterize the paths in the FLN as a matrix with each column corresponding to a path. In our sample network 
(Fig.~\ref{fig:Traversal Visits}), the path starting at article A is the 
first column in the traversal visits matrix. 
An entry of $1$ indicates the path contains a given article and 
$0$ indicates the path does not.
To compute the number of traversal visits for an article, we sum the corresponding row in our matrix.
The traversal visits matrix for 
our Wikipedia dataset consists of 121 million entries to account for each path through the FLN. 

By measuring the number of first links between
two articles, we obtain an additional piece of information we call 
{\it path length.} We can compute the length of a path by summing 
along a column in our traversal visits matrix. 
In our Wikipedia dataset, the sum of all path lengths is 232 million first links.
The path length describes how closely related topics are.
The greater the number of first links separating two articles, the greater the number of ideas needed to relate their corresponding ideas. Although
``Train'' is related to ``Economics'' for example, there are several articles
bridging the connection:
``Train'' is more specifically related to transportation, whose object
is often goods. Goods are one of the fundamental objects of study in Economics.
Described in links, this relationship is captured by a relatedness of $4$ 
first links ultimately connecting ``Train'' to ``Economics.'' 

One possible path through the FLN is a {\it cycle} or a group of articles 
linking to one another inside a loop. In our sample network 
(Fig.~\ref{fig:Traversal Visits}) 
a 3-cycle exists among nodes A, B, and C. 
We can readily identify
the types of cycle structures within the FLN
and rank each by 
the number of references directed towards a cycle. 


We can also form and rank {\it basins} in the network by identifying groups of 
path-connected articles, not necessarily forming a perfect cycle.
A basin connects a group of articles and identifies the paths 
to a particular article.

While traversal visits measure accumulation, each article's first link also 
influences the shape of the FLN. 
At a large point of accumulation, a single article's first link 
can exert great influence over the shape of the FLN by directing many
references on a particular path. To distinguish between an article 
that simply happened to fall within a cycle from an article funneling 
many first links, we develop a second metric called {\it traversal funnels}.

To measure traversal funnels, we traverse the FLN in the same manner as we 
did for traversal visits, but end a path once we enter a cycle.
We are then able to distinguish between an article related to many other ideas
only by virtue of its place in a cycle, from an article exerting influence over where the first links flow. 
An article with a large number of traversal funnels directs many references
into a particular cycle. In our sample network 
(see Fig.~\ref{fig:Traversal Funnels}) article C 
directs the flow of links towards the 3-cycle, while articles A and B are 
recipients of the flow---without exerting any direct influence themselves. 

By studying the FLN not only as collection of directly linked pairs of articles, but
as a flow, 
we build a powerful arsenal of information.
From accumulated references, cycles, and basins, we can measure how the many articles in Wikipedia are organized and related.





\section{Results}

\subsection{Degree Distribution}

\begin{figure}[tp!]
  \includegraphics[width=\columnwidth]{graphics/articles_ndegree.png}
  \caption{
    \textbf{Highest Ranking Articles by in-degree.}
We rank each article by the number of direct first links to the article (in-degree). The highest-ranking articles tend to represent geographical and biological abstractions.}
  \label{fig:indegree list}
\end{figure}

We rank all 11 million articles by in-degree to find 
the ``United States'' with 80,249 direct first links as the most referenced
Wikipedia article 
(Fig.~\ref{fig:indegree list}). 
Other high-ranking articles
include foundational abstract concepts such as ``village,'' ``species,''; 
sports associations such as ``American Football,'' ``Association Football''; 
and developed nations such as ``France,'' ``Japan,'' ``Russia,'' ``Australia,'' and 
the ``Netherlands.'' These high ranking articles are useful abstractions: nations
describe a collection of individuals with a common culture, language, or 
geographical proximity; sports teams describe an ever changing collection of 
sports players often associated with a cultural identity or a geographical 
region. 
Since abstractions such as nations and teams are inherently comprised
of many parts, authors reference the abstraction when describing a part.
In describing the New England Patriots or the New York Giants
for example, ``American Football'' is the natural abstraction, which 
anchors many specific teams.

``Philosophy'' and other philosophical concepts
are not among the highest-ranking articles by in-degree.
``Philosophy'' has an in-degree of only 581, with direct first links from articles about Philosophers and areas of Philosophy: ``Existentialism and Humanism,'' ``Predeterminism,'' ''Synoptic Philosophy,'' ``Qualia,'' ``Dorothy Emmet,'' and ``Christopher W. Morris.''
While many articles accumulate at ``Philosophy'' (see traversal visits discussion below), 
the accumulation is not the 
result of many articles directly referencing ``Philosophy.'' 
Instead, first links flow towards ``Philosophy'' as the 
ultimate anchor, by generalizing from specific articles to broader notions.

\begin{figure}[tp!]
  \includegraphics[width=\columnwidth]{graphics/ndegree_loglog.png}
  \caption{
    \textbf{FLN Degree Distribution.}
We fit the in-degree distribution with a linear $\log_{10}-\log_{10}$ model. 
The result is a 
an excellent fit with a Pearson's Correlation Coefficient of $-0.98$, yielding a 
power law rank exponent $\alpha = -0.788$ and a size exponent $\gamma = -0.266$. 
}
  \label{fig:degree distribution}
\end{figure}

The FLN's in-degree exhibits a decaying power law distribution where a few articles 
receive most direct first references, while most articles receive few or none.
The average in-degree for all 11 million articles is 3.6 direct first links with a standard deviation of 89.5 links.
Less than $1\%$ of articles have more than 100 direct first links and $75\%$ of articles
have fewer than 9. 
We fit the in-degree distribution against each article's rank with a log-log (base 10) linear regression model.
We obtain a Pearson's Correlation Coefficient of $-0.98$
with a corresponding  power law rank exponent $\alpha = -0.788$ and a size exponent $\gamma = -0.266$. 


\subsection{Depth of the FLN}

We first seek to describe the depth of the FLN: how many links does a 
connection among articles span? 
We find the longest path length is 365,
corresponding to the yearly calendar of Orthodox Liturgics.
Each day's Liturgics links to the next day's. On the last calendar day, the last article simply links back to January 1, forming a 365-cycle 
(see discussion of cycles).
We also find similarly lengthy paths following the evolution of a place or topic through time: 
``1953 in Scotland'' or ``1560s Architecture,'' with articles sequentially proceeding by year, decade or era.
In general, the longest paths connect temporally organized ideas.

\begin{figure}[tp!]
  \includegraphics[width=\columnwidth]{graphics/path_lengths_dist.png}
  \caption{
    \textbf{Path Length Distribution.}
The network depth appears to have a bimodal distribution with a median of 29 first links in a path.
The 365 cycle for ``Orthodox'' liturgics is the outlier to the right, with historical articles 
describing nations such as ``Scotland'' and ``UK'' also lying outside the third quartile.
More than $75\%$ of articles have path lengths between 
$0$ and $50$ links.}
  \label{fig:Path Length Distribution}
\end{figure}

Of the 11 million articles, 5.5 million had an invalid link or linked back to the same article, yielding a path length of zero. 
This roughly corresponds to the official number of articles on Wikipedia: 
$~4.7$ million as of November 2014---approximately half of the 11 million 
articles in the XML dump are redirects or disambiguations, not full articles.
The most common path length is 29, with an interquartile range of 4---path length
is typically between 26 and 29 articles.
As a distribution, more than $75\%$ of articles have a path length below 
$50$ first links 
while a few temporally organized paths exceed 50 links 
(see Fig.~\ref{fig:Path Length Distribution}). 



\subsection{Traversal Visits}

As a distribution, the number of traversal visits by article appears to follow a decaying power law. 
The majority of articles have fewer than 30 traversal visits and
first link references accumulate at a few articles.
Specifically, $99.76\%$ of articles have fewer than $100$ traversal visits; nearly $80\%$ have none. 
Meanwhile, the highest ranking 30 articles have an extremely disproportionate number of traversal visits.

\begin{figure}[tp!]
  \includegraphics[width=\columnwidth]{graphics/traversals_per_article.png} 
  \caption{
    \textbf{Distribution of Traversal Visits.}
We fit the distribution of traversal visits against article rank with a log-log (base 10) linear regression model log-log model and uncover two regimes: 
with a cutoff rank of $10^{5}$.
The top regime ($\log_{10}$(rank) $< 5$) has a Pearson's Correlation 
Coefficient of $-0.99$ with a corresponding power law rank exponent 
$\alpha = -1.23$ and a size exponent $\gamma = 0.187$.
The horizontal flattening around the highest
ranking articles is a result of the cyclic structure (see discussion on cycles).}
  \label{fig:Distribution of Visits}

\end{figure}

In Fig.~\ref{fig:Distribution of Visits}, we show a log-log (base 10) plot for the entire dataset: traversal visits against rank. 
We fit the data to a two regime log-log linear regression with a cutoff rank 
of $10^{5}$.
The top regime ($\log_{10}$(rank) $< 5$) has a Pearson's Correlation 
Coefficient of $-0.99$ with a corresponding power law rank exponent 
$\alpha = -1.23$ and a size exponent $\gamma = 0.187$.
On the aggregate, the distribution suggests
a handful of the highest ranking articles contain a disproportionate number of traversal visits, while most have none. The skew in the distribution is not terribly surprising when considering the heuristic of how the links flow: from specific to general. 


\begin{figure}[tp!]
  \includegraphics[width=\columnwidth]{graphics/articles_ranked.png}
  \caption{
    \textbf{Highest ranking articles by number of traversal visits.}
We compute the number of traversal visits for each article in the FLN (see 
Traversal Algorithm section for details). In doing so, we can rank each article
by the accumulation of first links. Articles with a greater number of traversal visits
mark greater points of first link accumulation. The highest ranking articles by traversal visits reveal where the greatest accumulation occurs.}
  \label{fig:highest visits}
\end{figure}



\subsection{Network Cycles}

\begin{figure*}[tp!]
  \includegraphics[width=\textwidth]{graphics/2_cycles.pdf}
  \caption{
    \textbf{Highest ranking 2-cycles.}
We identify pairs of articles whose first links point to one another, forming
a 2-cycle. We then rank each pair of articles by the total number of 
traversal visits to gauge the most referenced groups of two articles linked
to each other. The area of the circles in the figure indicates the number of traversal visits. We find 2-cycles often capture synonyms or articles representing nearly the 
same concepts as opposed to distinct concepts.}
  \label{fig:2-cycles}
\end{figure*}
We first identified 2-cycles, meaning a pair of articles with first link pointing to one another.
Of the 11 million articles, 84,000 are 2-cycles. 
The highest ranking 2-cycles by traversal visits tend to be synonyms (or nearly so) rather than distinct, yet connected topics:
``Health Care'' and ``Medical Case Management,'' ``Broadcasting House,'' and ``BBC,'' ``Secondary Education'' and ``Secondary School'' 
(see Fig.~\ref{fig:2-cycles}).

\begin{figure*}[tp!]
  \includegraphics[width=\textwidth]{graphics/3_cycles.pdf}
  \caption{
\textbf{Highest ranking 3-cycles.}
We identify groups of three articles whose links form a 3-cycle with the 
last article in the group linking back to the first article.
We rank each 3-cycle by the total number of traversal visits to gauge
the most referenced group of three linked articles. 
The area of the circles in the figure indicates the number of traversal visits.}
  \label{fig:3-cycles}
\end{figure*}

Outside of the highest ranking 2-cycles, the typical 2-cycle signals a connection between distinct, yet very closely related concepts. 
We also observe link patterns such as inventor to product (``Voere'' to ``VEC-91''), event to organizer (``Poetry Bus Tour'' to ``Weave Books''), and book to author (``Anatomy of Britain'' to ``Anthony Sampson'').

Similarly, 3-cycles capture a synonymous or close relation among 3 articles: ``Tree of life (Biology),'' ``Tree of life (disambiguation),'' 
and ``Tree of life''; ``Cinema of India,'' ``Indian Cinema,'' and ``Telugu Cinema''
(see Fig.~\ref{fig:3-cycles}).
Once we extend our cycle size beyond a length of 6 however, 
``Philosophy'' along with the remaining list of high ranking articles by traversal visits dominate.
The longest cycle in the network spans 365 articles of Eastern Orthodox Liturgics for each calendar day.
Other lengthy cycles span $60-75$ articles including collections of articles on national histories such as ``Japanese Eras'' 
or judicial bodies such as the ``Legislative Assembly of Ontario.''







\subsection{Basins}

We can group articles lying on the same path to identify {\it basins}: 
a group of path-connected articles.
Since cycles identify only groups of articles with a closed set of links, 
we additionally measure and rank basins to capture groups of closely related
articles branching outside of a cycle into the rest of the FLN.
We rank basins by the total number of traversal visits for each article in the path. 
Akin to river networks, these basins are areas of accumulation with a path 
flowing outwards to the rest of the FLN.

The highest ranking basins by the number of traversal visits are groups of articles
around ``Philosophy.'' 
The highest ranking paths include branches of philosophy flowing through 
``Awareness,'' ``Existence,'' and ``Consciousness'' to ``Philosophy.'' Other paths
include concepts around ``Mathematics,'' ``Scientific,'' ``Experiments,'' 
``Biology,'' and ``Fact.''
These paths link many specific articles to ``Philosophy,'' each funneled through a particular domain.

Excluding basins around ``Philosophy,'' we find other basins around 
foundational concepts such as ``Community,'' ``Landmass,'' ``Federal Government,'' 
``Presentation,'' and ``Belief System.'' 
The basins around each of these foundational notions are 
various paths containing related articles. For example around 
``Community'' we find basins
flowing from ``United States'' to ``Federal Republic'' to ``Political Union'' to ``State'' culminating at ``Community''; we also find basins flowing from 
``Public Policy'' to ``Executive (government)'' to ``Government'' to ``State'' and then 
to ``Community''; we also find paths flowing through a similar chain beginning
with ``Democracy,'' another beginning with ``Constitution,'' another at 
``Dictatorship,'' and so on. The articles build from specific means of organizing
a community (or society) and then build up to ``Community.'' 
Other basins around landmass for example begin at specific geographical regions
such as ``Eastern Europe'' building up to ``Continent'' and finally ``Landmass''.


\subsection{Traversal Funnels}

\begin{figure}[tp!]
  \includegraphics[width=\columnwidth]{graphics/top_funnels.png}
  \caption{
    \textbf{Funnels.}
We represent the highest-ranking articles by the number of traversal 
funnels to gauge the influence each article exerts in shaping the 
structure of the FLN. 
The area of each circle is sized in proportion to the number of 
traversal funnels each article has. 
We find ``Philosophy'' exerts an overwhelming proportion
of the influence, with other abstract notions and topical concepts ranking
next.}
  \label{fig:Funnels}
\end{figure}
To analyze the influence an article exerts in shaping the 
structure of the FLN, we compute the number of traversal funnels for each.
Articles directing more paths exert a greater influence over the structure
of the FLN by increasing the accumulation of first links
on a particular path. By measuring traversal funnels, we distinguish between an article that simply happened to fall within a cycle from an article funneling 
many first links.

\begin{figure}[tp!]
  \includegraphics[width=\columnwidth]{graphics/funnels_distribution.png}
  \caption{
    \textbf{Distribution of Traversal Funnels.}
We fit the distribution with a linear log-log model by considering the log (base 10) transformed rank of each article against log (base 10) transformed the number of traversal funnels. 
We find two regimes with the top regime (log(rank) < 4) 
well-explained by a linear fit, yielding Pearson's 
Correlation Coefficient of $-0.99$ and a 
corresponding power law rank exponent $\alpha = -1.08$
and a size exponent $\gamma = 0.074$. The top regime 
explains the distribution of traversal funnels for the $17, 821$ 
articles with at least one traversal funnel. The bottom 
horizontal regime corresponds to the more than $99\%$ of articles
which hold zero traversal funnels.}
  \label{fig:Funnels Distribution}
\end{figure}
Ranking articles by the number of traversal funnels we find 
``Philosophy'' to be by far the highest-ranking article with 
$7.37$ million paths
(see Fig.~\ref{fig:Funnels}).
Of any article, the number of traversal funnels Philosophy holds exceeds 
all others by more than two orders of magnitude.
The ``Philosophy'' cycle which contains ``Existence,'' ``Awareness,'' ``Reality,'' 
and similar articles accumulates the overwhelming proportion of its 
references through ``Philosophy'': $7.37$ million of the $7.4$ million references
are funneled through ``Philosophy''.
Second on the list of highest-ranking articles by traversal funnels is 
``Presentation'' with only $30$ thousand paths. Similarly abstract 
concepts also rank highly such as ``Tree of life'' (30 thousand), 
``Reality'' (13 thousand), and ``Jurisdiction'' (3 thousand).

Many high-ranking articles are remarkably topical, culturally and politically important concepts.  For example, ``Health Care,'' a recently high-contested legislative topic appears high on the list---Google trends indicates an uncharacteristic spike in search frequency between August, 2009 and February, 2010.
Other high ranking articles include key historical events such as the ``Cold War'' or critical scarce resource with recent 
media discussion such as ``Fossil Fuel.'' 
The highest-ranking list also includes ``Hip Hop,'' ``Cancer,'' and ``Web Page.''

As a distribution, we find few articles influence the structure of the 
FLN. Only $17, 821$ articles have one or more traversal funnels, leaving
an more than $99\%$ with none---most articles are recipients of 
the references flowing through the articles with at least one traversal funnel.
When fit to a log-log linear model we find the $99\%$ of articles with zero
traversal funnels form one regime (with log(rank) less than 4).
The top regime, corresponding to the $17, 821$ articles with at least one 
traversal funnel strongly fits a linear model with
a Pearson's Correlation Coefficient of $-0.99$ and a 
corresponding power law rank exponent $\alpha = -1.08$
and a size exponent $\gamma = 0.074$. Even within the few articles
influencing the structure of the FLN, only a handful of these exert most of the 
control. 

\subsection{Article Popularity}

\begin{figure}[tp!]
  \includegraphics[width=\columnwidth]{graphics/views_visits.png}
  \caption{
\textbf{Article Popularity (by page views) compared to Traversal Visits 
for Highest ranking 1000 articles.}
We use the total number of page views provided by Wikipedia for the month
of October 2015 to compare each article's popularity to traversal visits.
While the most popular articles do not necessarily correspond to the articles
with the highest number of traversal visits, the variation in popularity appears to decrease as the number of traversal visits increases. The greatest number of articles fall within $10^{3.8}-10^5$ traversal visits and page views.}
  \label{fig:Views and Visits}

\end{figure}
Wikipedia released an API to measure article popularity by page views
starting November 2015. Page views add another dimension to our
findings by revealing 
how users view the information on Wikipedia. 
We measure popularity as the total number 
of page views in the English edition of Wikipedia in the month of 
October 2015---the earliest full month for which the data is available. 
We find the highest ranking 1000 articles (by traversal visits) have an average of
$70$ thousand page views in October with high variation: the standard deviation 
is $1.1$ million page views. 
The number of page views has a skewed distribution with $75\%$ of articles
reaching fewer than $73$ thousand page views in October.
The article for ``United States'' has the most page 
views of the highest ranking $1000$ articles by traversal visits with 
$1.4$ million views in October. The next most popular articles are 
``India,'' ``World War II,'' and the ``United Kingdom'' each with $0.7-1.03$ million page views. 


\begin{figure}[tp!]
  \includegraphics[width=\columnwidth]{graphics/funnels_visits.png}
  \caption{
    \textbf{Article Popularity (by page views) compared to Traversal Funnels
    for Highest ranking 1000 articles.}
We use the total number of page views provided by Wikipedia for the month
of October 2015 to compare each article's popularity to traversal funnels.
In the top left is ``Halloween'' with a significantly greater 
number of page views in the month of October. 
The greatest number of articles falls within $10^3-10^5$ page views, 
but only $10^{1.2}-10^2$ traversal funnels.}
  \label{fig:Views and Funnels}
\end{figure}
The ``Philosophy'' article, despite outranking every article by traversal visits,
has only $240$ thousand page views in October compared to the $1.4$ million
page views for ``United States''.




We also analyze article popularity for the highest ranking 1000 articles by 
traversal funnels. In October, the average page views per article is 
$22$ thousand with a standard deviation of $95$ thousand views. The distribution
is skewed with $75\%$ of articles reaching fewer than $20$ thousand page views. 
``Halloween'' is the most popular article with $2.8$ million views in October,
likely a result of Halloween falling in the month of October. 
Other popular articles include ``Scientology'' ($463$ thousand views), ``Clint Eastwood'' 
($341$ thousand views),
and the ``Cold War'' ($298$ thousand views) although each has significantly fewer views compared to the views for ``Halloween''.
Other standout popular October articles include ``24-hour'' and ``12-hour'' clocks likely due
to Daylight savings in October and ``Marriage'' likely due to the drop in the number of
weddings in the months following October
\cite{weddings}.
``Philosophy'' which ranks seventh among the most popular of the highest ranking articles by traversal funnels, appears no where in the top 20
by traversal visits. ``Philosophy'' is a relatively popular article 
among articles influencing the shape of the network, but less popular among 
highly-ranked articles by accumulation.





%========================Conclusion====================================

\section{Concluding Remarks}

The findings here should only be considered within the limitations of their context.
We examined only the English editiion of Wikipedia at a particular moment in time, 
considering only the first link in in each article as a means to related one article to another.
Finally, Wikipedia, while the largest 
collection of human knowledge, is rife with the biases of the many contributing editors
\cite{bias}
. Nevertheless, the findings do reveal
generalizable relationships and point to foundational notions.


Among our observations is the multiple appearance of scale-free distributions within the network. 
Few articles have most traversal visits, few paths have an exceptionally long path length, and even fewer
articles are responsible for funneling most paths. When measured against the traversal funnels, 
``Philosophy'' emerges as an exceptional article by orders of magnitude. 
Nevertheless, many other foundational concepts emerged naturally within the First Link Network. 
Basins around ``Community'', ``State'', and ``Science'' reveal a foundational structure within the network. 
More curious is the emergence of recently prominent political and economics topics such as ``Fossil Fuel'' and ``Health Care'' 
within the highest ranking funnels. 
Wikipedia seems to reflect not only timeless foundations, but also the topical (at least within English speaking society).

Future work could examine other language versions of Wikipedia for potentially telling cultural or regional differences as well as expand the network to more than the First Link.
These findings also form the basis for the creation of a taxonomy where 
every idea, event, or object sits within a hierarchy of connected notions.
The taxonomy would extend a traditional word thesaurus beyond mere synonyms to a related hierarchy of concepts.
Applications could range from an enhanced network of ideas to psychological insights into how humans form associations.
Specifically, an ever-evolving reference of related hierarchical concepts could be applied to search engine algorithms 
or natural language processing.


\acknowledgments
MI is grateful to RJ for pointing out the Reddit post 
\cite{reddit}
describing how the majority of links lead to ``Philosophy,'' inspiring this research.


\newpage

\section{Appendix}

\subsection{Constructing The First Link Network}

To map Wikipedia's First Link Network, we use the freely-available XML dump of the English edition of Wikipedia. 
Rather than rely on a sample of articles from which to generalize, we opted to process the entirety of Wikipedia, 
eliminating any statistical error due to sampling.
We analyze the snapshot provided on November 2014, representing the state of Wikipedia at the time.
The November raw dump consists of 11 million articles: 4.7 million unique articles along with redirects
and disambiguations.
Knowing Wikipedia is an ever-evolving project with 10 edits every second and 750 new articles per day on average
\cite{wiki_edits},
our aim is to characterize the dynamics of the First Link Network, not record a particular link between one
article and another.

Wikipedia renders and stores articles in MediaWiki markup, a markup language with syntax and keywords to format and mark elements in a page. Along with special syntax for links, MediaWiki markup includes templates for audio files, images, and side-bar
information.
While a human could accurately identify the First Link, to map the entire First Link Network of 11 million articles, we needed to programmatically untangled the body text from side-bar, header box, and invalid link elements.

While some libraries exist for MediaWiki Markup,
Approaches using existing libraries led to several bugs 
including trouble with nested links, nested parenthesis, unclosed tags, escape characters 
as well as compatibility with other libraries used to parse the XML.
Consequently,  we developed an algorithm for parsing the First Link in the XML version of each article.
Our parsing algorithm aimed to: 
1) accurately identify the First Link among other page elements, and 
2) efficiently do so---that is without needing for several passes through the data.
To process an article in a single pass, we developed a hierarchical system of flags:
\begin{figure}[tp!]
  \includegraphics[width=\columnwidth]{graphics/flags.pdf}  
  \caption{
    \textbf{Parsing Algorithm of Wikipedia's XML dump.}
The highest flag in the hierarchy indicates a Wikimedia template used to mark an element in the side bar, display an image, link to an external file, or another Wikimedia project outside of Wikipeida. Next, to catch any remaining elements outside the main body we have a second flag for <ref>, <div> elements. Finally, we identify parenthesis to ensure we do not capture a link to a pronunciation key.}
  \label{fig:parsing algorithm}
\end{figure}


The algorithm loops in three-character chunks to account for potentially nested elements, 
shifting by one character steps through the article markup.
If any markup triggers for a flag are detected, a flag is raised. 
Once a flag is raised, we stop processing and proceed to the next character
until the flag's closing markup.
A First Link is identified only if Flags 1, 2, and 3 are all off.
In this case, the entire link is retrieved. 
We then confirm the link is valid by filtering for MediaWiki keywords indicating external page or other projects
as well as common file extensions for 
images, audio files, and the like 
\cite{media_wiki_templates}.
The First Link of an article is then the earliest valid link with unraised flags.

To process the entirety of Wikipedia, we distributed the parsing and processing of the XML dump
across 112 cores of the UVM supercomputer cluster
\cite{vacc}
We then joined the results to form a hash table containing every Wikipedia article and its corresponding
First Link. The resulting network map is the basis of our analysis.





\begin{thebibliography}{1}

      \bibitem{editors} Iba, Takashi, et al. ``Analyzing the creative editing behavior of Wikipedia editors: Through dynamic social network analysis.'' Procedia-Social and Behavioral Sciences 2.4 (2010): 6441-6456.

      \bibitem{accuracy1} Holman Rector, Lucy. ``Comparison of Wikipedia and other encyclopedias for accuracy, breadth, and depth in historical articles.'' Reference services review 36.1 (2008): 7-22.

      \bibitem{accuracy2} Giles, Jim. ``Internet encyclopaedias go head to head.'' Nature 438.7070 (2005): 900-901.

       \bibitem{coverage} Halavais, Alexander, and Derek Lackaff. ``An analysis of topical coverage of Wikipedia.'' Journal of Computer‐Mediated Communication 13.2 (2008): 429-440.

        \bibitem{bias_women} Hill, Benjamin Mako, and Aaron Shaw. ``The Wikipedia gender gap revisited: characterizing survey response bias with propensity score estimation.'' PloS one 8.6 (2013): e65782.

      \bibitem{missing_entries} Williams, Jake Ryland, Clark, Eric M., Bagrow, James P., Danforth, Christopher M. \& Dodds, Peter Sheridan (2015). Identifying missing dictionary entries with frequency-conserving context models. Phys. Rev. E, 92, 042808.

      \bibitem{clustering} Banerjee, Somnath, Krishnan Ramanathan, and Ajay Gupta. ``Clustering short texts using wikipedia.'' Proceedings of the 30th annual international ACM SIGIR conference on Research and development in information retrieval. ACM, 2007.

      \bibitem{semantic_relatedness} Gabrilovich, Evgeniy, and Shaul Markovitch. ``Computing Semantic Relatedness Using Wikipedia-based Explicit Semantic Analysis.'' IJCAI. Vol. 7. 2007.

      \bibitem{disambiguating} Cucerzan, Silviu. ``Large-Scale Named Entity Disambiguation Based on Wikipedia Data.'' EMNLP-CoNLL. Vol. 7. 2007.

      \bibitem{drugs} Clauson, Kevin A., et al. ``Scope, completeness, and accuracy of drug information in Wikipedia.'' Annals of Pharmacotherapy 42.12 (2008): 1814-1821.

      \bibitem{stats} Wikipedia contributors. ``Wikipedia:Statistics.'' Wikipedia, The Free Encyclopedia. Wikipedia, The Free Encyclopedia, 9 Nov. 2015. Web. 13 Nov. 2015.

      \bibitem{pagerank} Page, Lawrence, et al. ``The PageRank citation ranking: bringing order to the Web.'' (1999).

      \bibitem{classifier} Chakrabarti, Soumen, Mukul Joshi, and Vivek Tawde. ``Enhanced topic distillation using text, markup tags, and hyperlinks.'' Proceedings of the 24th annual international ACM SIGIR conference on Research and development in information retrieval. ACM, 2001.

      \bibitem{relevance} Kamps, Jaap, and Marijn Koolen. ``Is Wikipedia link structure different?.'' Proceedings of the second ACM international conference on Web search and data mining. ACM, 2009.
          
      \bibitem{wiki_train} Wikipedia contributors. ``Train.'' Wikipedia, The Free Encyclopedia. Wikipedia, The Free Encyclopedia, 
          \url{https://en.wikipedia.org/wiki/Train}

    \bibitem{locke} Bolton, Martha Brandt. ``The Taxonomy of Ideas in Locke’s {\it Essay},'' The Cambridge Companion to Locke's 'Essay Concerning Human Understanding'. Ed.Lex Newman.. 1st ed. Cambridge: Cambridge University Press, 2007. 67-100. Cambridge Companions Online. Web. 13 November 2015. http://dx.doi.org/10.1017/CCOL0521834333.004.

    \bibitem{aristotle} Studtmann, Paul, ``Aristotle's Categories,'' The Stanford Encyclopedia of Philosophy (Summer 2014 Edition), Edward N. Zalta (ed.), \url{http://plato.stanford.edu/archives/sum2014/entries/aristotle-categories/}

    \bibitem{descartes} Smith, Kurt, ``Descartes' Theory of Ideas,'' The Stanford Encyclopedia of Philosophy (Spring 2014 Edition), Edward N. Zalta (ed.), \url{http://plato.stanford.edu/archives/spr2014/entries/descartes-ideas/}


    \bibitem{hist_thesaurus} What is the Historical Thesaurus of the OED - Oxford English Dictionary (Oxford English Dictionary)
    http://public.oed.com/historical-thesaurus-of-the-oed/what-is-the-historical-thesaurus-of-the-oed/

    \bibitem{vacc} The authors acknowledge the Vermont Advanced Computing Core which is supported by NASA (NNX 06AC88G), at the University of Vermont for providing High Performance Computing resources that have contributed to the research results reported within this paper.
    \url{http://www.uvm.edu/vacc}

    \bibitem{wiki_views} Page Views for Wikipedia, Non-mobile site, Normalized (Page Views for Wikipedia, Non-mobile site, Normalized)
    \url{https://stats.wikimedia.org/EN/TablesPageViewsMonthly.htm}

    \bibitem{weddings} Wedding statistics in the United States (Wedding statistics in the United States)
    \url{http://www.soundvision.com/article/wedding-statistics-in-the-united-states}

    \bibitem{six_degrees} Elmacioglu, Ergin, and Dongwon Lee. ``On six degrees of separation in DBLP-DB and more.'' ACM SIGMOD Record 34.2 (2005): 33-40.

    \bibitem{wordnet} Princeton University ``About WordNet.'' WordNet. Princeton University. 2010. 
    \url{http://wordnet.princeton.edu}

    \bibitem{newman} Newman, Mark EJ. ``The structure and function of complex networks.'' SIAM review 45.2 (2003): 167-256.

    \bibitem{disease} Eubank, Stephen, et al. ``Modelling disease outbreaks in realistic urban social networks.'' Nature 429.6988 (2004): 180-184.

    \bibitem{social_nets} Newman, Mark EJ, Duncan J. Watts, and Steven H. Strogatz. ``Random graph models of social networks.'' Proceedings of the National Academy of Sciences 99.suppl 1 (2002): 2566-2572.

    \bibitem{geo_basins} Horton, Robert E. ``Erosional development of streams and their drainage basins; hydrophysical approach to quantitative morphology.'' Geological society of America bulletin 56.3 (1945): 275-370.
    APA

    \bibitem{dodds} Dodds, Peter Sheridan, and Daniel H. Rothman. ``Unified view of scaling laws for river networks.'' Physical Review E 59.5 (1999): 4865.

    \bibitem{food_webs} Garlaschelli, Diego, Guido Caldarelli, and Luciano Pietronero. ``Universal scaling relations in food webs.'' Nature 423.6936 (2003): 165-168.
    
    \bibitem{mat_blog} brain of mat kelcey (brain of mat kelcey). 
    ``Do all first links lead to Philosophy?''
    \url{http://matpalm.com/blog/2011/08/13/wikipedia-philosophy/}

    \bibitem{Ilmari_first_links} User:Ilmari Karonen/First link (Wikipedia)
    \url{https://en.wikipedia.org/wiki/User:Ilmari_Karonen/First_link}

    \bibitem{xkcd} xkcd: Extended Mind (xkcd: Extended Mind)
    \url{http://xkcd.com/903/}

    \bibitem{reddit} Almost every Wikipedia page links to Philosophy if you keep clicking the first link in every article. 
    \url{https://www.reddit.com/r/InternetIsBeautiful/comments/35asgt/almost_every_wikipedia_page_links_to_philosophy/}

    \bibitem{bias} Wagner, Claudia, et al. ``It's a Man's Wikipedia? Assessing Gender Inequality in an Online Encyclopedia.'' arXiv preprint arXiv:1501.06307 (2015).

    \bibitem{wiki_edits} Wikipedia:Statistics 
    \url{https://en.wikipedia.org/wiki/Wikipedia:Statistics}

    \bibitem{media_wiki_templates} MediaWiki: ``Help: Templates''
    \url{https://www.mediawiki.org/wiki/Help:Templates}

    \bibitem{links_that_speak} Ronen, Shahar, Bruno Gonçalves, Kevin Z. Hu, Alessandro Vespignani, Steven Pinker, and César A. Hidalgo. ``'Links That Speak: The Global Language Network and Its Association with Global Fame.'' Proceedings of the National Academy of Sciences Proc Natl Acad Sci USA 111.52 (2014).

\end{thebibliography}


\end{document}
