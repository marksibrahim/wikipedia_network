Apples, porcupines, and the most obscure Bob Dylan song---is every topic a few clicks from Philosophy? 
Within Wikipedia, the surprising answer is yes: nearly all 
paths lead to Philosophy.
Wikipedia is the largest, most meticulously indexed collection of human knowledge ever amassed. 
More than information about a topic, Wikipedia is a web of naturally emerging relationships.  
By following the first link in each article, we algorithmically construct a directed network of 
all 4.7 million articles: Wikipedia's First Link Network.
Here, we study the English edition of Wikipedia's First Link Network for insight into how the many 
articles on inventions, places, people, objects, and events are related and organized.  


By traversing every path, we measure the accumulation of first links, path lengths,
groups of path-connected articles, and cycles.
We also develop a new method, traversal funnels, to measure the influence each article exerts in shaping the network. 
Traversal funnels provides a new measure of influence for directed networks without spill-over into cycles, in contrast to traditional network centrality measures.
Within Wikipedia's First Link Network, we find scale-free distributions describe path length, 
accumulation, and influence. Far from dispersed, first links disproportionately accumulate 
at a few articles---flowing from specific to general and culminating around fundamental notions such as
Community, State, and Science. 
Philosophy directs more paths than any other article by two orders of magnitude. 
We also observe a gravitation towards topical articles such as 
Health Care and Fossil Fuel. 
These findings enrich our view of the connections and structure of
Wikipedia's ever growing store of knowledge.

