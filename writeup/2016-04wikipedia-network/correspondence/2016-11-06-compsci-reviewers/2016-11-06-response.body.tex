\textbf{We thank the reviewers for their well-considered comments and
suggestions.}

\textbf{As we detail below, we have worked hard to address all concerns in full,
  and we believe our manuscript has been strengthened.}


\section*{Reviewer 1}

%% \reviewercomment{1}{}
%% \reply{}

\reviewercomment{1}{
I am not fully satisfied with the updates that the authors have provided. After taking into account the comments by Reviewer 2, many of which are valid concerns, I would like to see another revision addressing the following two points.

(a) Present in an algorithm form,  the steps that the authors used to compute the "Traversal-Funnel" based influence measure instead of a descriptive discussion. Comment on the running time complexity of the same.

(b) Please double check the eigenvector centrality values. Normally the eigenvector corresponding to the top eigenvalue is selected and the entires are all non-negative. There are some negative entries present in the values shown.
}


\reply{In response to the reviewer's suggestions, we:

(a) Included pseudo-code for the traversal funnels algorithm as well as a description of the runtime and auxiliary memory usage in section B (Traversal Funnels). We have also made all code used in this work available on the online appendix.

(b) Re-rendered the Centrality Measures table (Fig. 4) to reflect the accurate, non-negative eigenvector centrality values. The previous submission contained incorrectly rendered scientific notation showing negative values. The true values included in our source code (see online appendix) are effectively zero:
 D: $-3.65*10^{-17}$, 
 F: $-3.81*10^{-17}$, 
 and 
G: $-8.26*10^{-18}$.
The Centrality Measures table (Fig 4) now correctly displays the true values.
}

\section*{Reviewer 2}

\reviewercomment{1}{
I am satisfied with the modifications made and the author response.  I would suggest adding something to the introduction pointing out that you are introducing the new notion of traversal funnels, and make it a bit more prominent that this is a novel measure you have created.
}

\reply{
In response to the reviewer's suggestion, we more prominently highlighted traversal funnels as new measure of influence in various sections including the abstract, introduction, and conclusion.
}
